%%%%%%%%%%%%%%%%%%%%%%%%%%%%%%%%%%%%%%%%%%%%%%%%%%%%%%%%%%%%%%%%%%
%%%%%%%% CS 4437 CS 9637 EXAMPLE LATEX SUBMISSION FILE %%%%%%%%%%%
%%%%%%%%%%%%%%%%%%%%%%%%%%%%%%%%%%%%%%%%%%%%%%%%%%%%%%%%%%%%%%%%%%

% Document class
\documentclass{article}

% For figures
\usepackage{graphicx} % more modern
%\usepackage{epsfig} % less modern
\usepackage{subfigure} 

% For citations
\usepackage{natbib}

% For algorithms
\usepackage{algorithm}
\usepackage{algorithmic}

% As of 2011, we use the hyperref package to produce hyperlinks in the
% resulting PDF.  If this breaks your system, please commend out the
% following usepackage line and replace \usepackage{cs4437cs9637} with
% \usepackage[nohyperref]{cs4437cs9637} above.
\usepackage{hyperref}

% Packages hyperref and algorithmic misbehave sometimes.  We can fix
% this with the following command.
\newcommand{\theHalgorithm}{\arabic{algorithm}}

% Employ the following version of the ``usepackage'' statement for
% submitting the draft version of the paper for review.  This will set
% the note in the first column to ``Under review.  Do not distribute.''
\usepackage{cs4437cs9637} 

\begin{document}

% The \cstitle you define below is probably too long as a header.
% Therefore, a short form for the running title is supplied here:
\cstitlerunning{Project Proposal}

\twocolumn[
\cstitle{Project Proposal}

% It is OKAY to include author information
\csauthor{}{}
\csaddress{The University of Western Ontario,
            London, Ontario. Canada}

% You may provide any keywords that you 
% find helpful for describing your paper; these are used to populate 
% the "keywords" metadata in the PDF but will not be shown in the document
\cskeywords{}

\vskip 0.3in
]


\section{Attack Description}\label{attack-description}

Since it's introduction in 2006, WPA2 has become the most popular
protected access technology used in the consumer industry. Almost every
household WiFi network is currently making use of WPA2 to authenticate
its devices. \citet{KrackAttacks}

On October 16th, 2017 a flaw was found in the WPA2 protocol, making
every device using it to connect to a Wi-Fi network vulnerable and
helpless to hackers. No matter the encryption method, from WPA2 to AES,
hackers are able to decrypt any data sent by the victim. In fact,
because this is a man in the middle attack, the device can be tricked
into using an all-zero encryption key. If no other form of encryption is
utilized (HTTPS), the attacker will have access to unencrypted and
human-readable data, even when it comes to sensitive information like
usernames, passwords, and credit-card information. If that isn't bad
enough, it is even possible to inject ransomware or malware into
websites. Meaning, this is not only bad for the users but also the web
providers. The worst part is that there is no way to stop this, other
than hardwiring the device, or hoping the manufacturer of the device
offers an update that patches this attack.

The method of using this weakness in Wi-Fi is called a Key
Reinstallation Attack (KRACK). Every modern Wi-Fi standard uses a 4-way
handshake to connect. This is where the server and client agree on an
encryption key to use when encrypting and decrypting all future
messages. The attack occurs in the third message of the 4-way handshake.
What happens is the hacker tricks the victim into reinstalling the
encryption key that is already in use. This is achieved through
manipulation and replaying of cryptographic handshake messages.
Theoretically, it should not be possible to install the same key
multiple times in a row, but because it is very common for messages sent
over Wi-Fi to get lost, many messages in this handshake may need to be
retransmitted. This makes the attack able to reset the nonce any time it
retransmits this third message, enabling packets to be replayed, then
decrypted and even forged. The whole purpose of this nonce was to ensure
that past massages could not be reused in replay attacks.

There are many variations of this attack that have been discovered that
have varying levels of impact. One of the worst and most devastating is
being able to reinstall an all-zero encryption key. Being the worst
variation one would hope it is not too common, however, all devices
using Linux or Android 6+ are susceptible. This all-zero encryption key
makes it easy to manipulate and intercept data sent at will since all of
this data is now readable text. With so many Linux and Android devices
present, this is especially scary and concerning. Even more so with
their only hope being to wait for a security update to fix this
vulnerability.

\section{The Goal}\label{the-goal}

Lorem ipsum dolor sit amet, consectetur adipiscing elit. Fusce porta
mauris sit amet finibus lacinia. Nunc id pharetra tortor, quis
scelerisque tellus. Nunc at est nec sapien tincidunt ultricies a quis
mi. Curabitur sed sem vitae ipsum varius molestie. Integer sed arcu
velit. Fusce ornare malesuada dolor, ut finibus arcu ornare ut. Nam
tincidunt sem in tempor pellentesque. Integer efficitur, nisl vel
euismod ultricies, nisl orci volutpat orci, nec ultrices nisi tortor id
eros.

\section{Proposed Solution}\label{proposed-solution}

\subsection{Description}\label{description}

Lorem ipsum dolor sit amet, consectetur adipiscing elit. Fusce porta
mauris sit amet finibus lacinia. Nunc id pharetra tortor, quis
scelerisque tellus. Nunc at est nec sapien tincidunt ultricies a quis
mi. Curabitur sed sem vitae ipsum varius molestie. Integer sed arcu
velit. Fusce ornare malesuada dolor, ut finibus arcu ornare ut. Nam
tincidunt sem in tempor pellentesque. Integer efficitur, nisl vel
euismod ultricies, nisl orci volutpat orci, nec ultrices nisi tortor id
eros.

\subsection{Technology Breakdown}\label{technology-breakdown}

Lorem ipsum dolor sit amet, consectetur adipiscing elit. Fusce porta
mauris sit amet finibus lacinia. Nunc id pharetra tortor, quis
scelerisque tellus. Nunc at est nec sapien tincidunt ultricies a quis
mi. Curabitur sed sem vitae ipsum varius molestie. Integer sed arcu
velit. Fusce ornare malesuada dolor, ut finibus arcu ornare ut. Nam
tincidunt sem in tempor pellentesque. Integer efficitur, nisl vel
euismod ultricies, nisl orci volutpat orci, nec ultrices nisi tortor id
eros.

\section{Mockups}\label{mockups}

Lorem ipsum dolor sit amet, consectetur adipiscing elit. Fusce porta
mauris sit amet finibus lacinia. Nunc id pharetra tortor, quis
scelerisque tellus. Nunc at est nec sapien tincidunt ultricies a quis
mi. Curabitur sed sem vitae ipsum varius molestie. Integer sed arcu
velit. Fusce ornare malesuada dolor, ut finibus arcu ornare ut. Nam
tincidunt sem in tempor pellentesque. Integer efficitur, nisl vel
euismod ultricies, nisl orci volutpat orci, nec ultrices nisi tortor id
eros.

\bibliography{Project Proposal}
\bibliographystyle{plainnat}

\end{document} 
